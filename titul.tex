\documentclass[12pt,a4paper]{scrartcl}
\usepackage[utf8]{inputenc}
\usepackage{fullpage}
\usepackage[english,russian]{babel}
\usepackage{indentfirst}
\usepackage{misccorr}
\usepackage{graphicx}
\usepackage{caption}
\usepackage{subcaption}
\captionsetup{compatibility=false}
\usepackage{amsmath}
\usepackage{wrapfig}
\usepackage{float}
\usepackage{verbatim}


\def\Rspace{\mathbb{R}}

\begin{document}


\begin{titlepage}
  \begin{center}

    	{ФЕДЕРАЛЬНОЕ ГОСУДАРСТВЕННОЕ БЮДЖЕТНОЕ ОБРАЗОВАТЕЛЬНОЕ УЧРЕЖДЕНИЕ ВЫСШЕГО ОБРАЗОВАНИЯ <<МОСКОВСКИЙ ГОСУДАРСТВЕННЫЙ УНИВЕРСИТЕТ имени М.В. ЛОМОНОСОВА>>}
    \vspace{0.4cm}
    
    {ФИЗИЧЕСКИЙ ФАКУЛЬТЕТ}
    \vspace{0.4cm}
    
    {КАФЕДРА МАТЕМАТИЧЕСКОГО МОДЕЛИРОВАНИЯ И ИНФОРМАТИКИ}
    \vspace{2.0cm}
    
	{БАКАЛАВАРСКАЯ РАБОТА}    
    \vspace{0.4cm}
    
    {\LARGE \textbf{<<КОМПЬЮТЕРНАЯ МОДЕЛЬ РЕМАСШТАБИРОВАНИЯ СЛОЖНЫХ СЦЕН ДЛЯ МОДЕЛЕЙ ПРОЦЕССОВ ПЕРЕНОСА>>}}
  \bigskip
    
\end{center}
\vfill

\hfill\begin{minipage}{0.4\textwidth}
	\begin{flushright}
	Выполнил студент\\
	435 группы\\
	Дейнека Даниил Андреевич\\
		\begin{minipage}{\textwidth}
		\vspace{0.8cm}
		\underline{\hspace{6.2cm}}\\
		\centering
		\small подпись студента
		\end{minipage}
	\end{flushright}
\end{minipage}%
\bigskip
\bigskip

\hfill\begin{minipage}{0.4\textwidth}
	\begin{flushright}
	Научный руководитель\\
	к.т.н., доцент Грачёв Е.А.\\
		\begin{minipage}{\textwidth}
		\vspace{0.8cm}
		\underline{\hspace{6.2cm}}\\
		\centering
		\small подпись научного руководителя
		\end{minipage}
	\end{flushright}
\end{minipage}

\bigskip
\vfill

\begin{minipage}{0.4\textwidth}
	\begin{flushleft}
	Допущена к защите <дата>\\
		\vspace{0.3cm}
		Зав.кафедрой\underline{\hspace{3.0cm}}\\
		\centering
		\small подпись зав.кафедрой
	\end{flushleft}
\end{minipage}


\vfill


\begin{center}
  Москва\\
  2018 г.
\end{center}

\end{titlepage}

\tableofcontents

\newpage
\section{Введение}

В некоторых прикладных областях, таких как геологическая разведка и нефтедобыча, существует проблема предсказания некоторого свойства объекта в пространстве по известной точной информации о его локальной структуре — карте признака. Одной из подзадач данной проблемы является ремасштабирование карт признаков.

При переходе на большие масштабы часто возникает ситуация, когда необходимо состыковать несколько карт признаков, чтобы получить одну карту объекта большего размера. Одним из подходов для получения карт является их генерация с помощью нейросетевых методов — в этом случае различные карты признаков получаются независимо друг от друга, из-за чего при их стыковке можно наблюдать резкую границу между ними. В связи с этим данная работа ставит перед собой задачу склейки карт признаков.

\section{Математическая постановка задачи}

Даны две $n$-мерных карт признака, которые пересекаются между собой. Каждую карту можно рассматривать как функцию $n$ переменных, заданной на некоторой области определения. Введём следующие обозначения (см. Рис. \ref{notations}):

$S$ — замкнутое подмножество $\Rspace^n$, область определения одной карты;

$\Omega$ — замкнутое подмножество $S$, область пересечения данных карт;

$\delta\Omega$ — граница множества $\Omega$;

$f : \Rspace^n \longrightarrow \Rspace^1$ — скалярная неизвестная функция, определённая на $\Omega$;

$f^* : \Rspace^n \longrightarrow \Rspace^1$ — скалярная известная функция, определённая на $S$;



\begin{figure}[htbp]
  \centering  
  \includegraphics[width=0.8\textwidth]{images/notations.jpg}
  \caption{Интерполяция $f$ известной функции $f^*$ в области $\Omega$.}\label{notations}
\end{figure}

Одной из интерполяций функции $f$ в области $\Omega$ является мембранная интерполяция, определённая как решение задачи минимизации \cite{Perez}:

\begin{equation}
\min_f \int\int_\Omega |\nabla f|^2
\end{equation}
с учётом граничных условий Дирихле
\begin{equation}
f|_{\delta\Omega} = f^*_{\delta\Omega}
\end{equation}

Задача вполне определена, однако осталась не использована информация, полученная из накладываемых изображений. Чтобы учесть это, добавим в задачу минимизации некоторое векторное поле:

\begin{equation}\label{min}
\min_f \int\int_\Omega |\nabla f-\textbf{v}|^2
\end{equation}
\begin{equation}\label{dirichle}
f|_{\delta\Omega} = f^*_{\delta\Omega},
\end{equation}
где $\textbf{v}$ — так называемый вектор \textit{направлений}, который в простейшем случае выбирается равным градиенту одного из изображений. Выбор конкретного вида $\textbf{v}$ будет подробно обсуждён в разделе~\ref{sec:vec}.

Поставленная задача является $n$-мерной вариационной задачей первого порядка. Рассмотрим обобщённый случай такой задачи:
\begin{equation}
	\begin{cases}
		J(f)=\int_\Omega L(x_1,\ldots,x_n,f,f_{x_1},\ldots,f_{x_n})dx_1 \ldots dx_n \\
		f = g, \; (x_1,\ldots,x_n) \in \delta\Omega \\
		\tilde f = \arg\min_f J(f),
	\end{cases}
\end{equation}

где $f_{x_i} = \frac{\partial f}{\partial x_i}, \; i=\overline{1,n}$ и $g$ — некоторая непрерывная функция.

Согласно \cite{Springer}, если функция $f$ является решением многомерной обобщённой вариационной задачи, то она также удовлетворяет уравнению Эйлера-Лагранжа на $\Omega$:
\begin{equation}\label{Euler}
\frac{\partial L}{\partial f} - \sum_{i=1}^n \frac{\partial L_{f_{x_i}}}{\partial x_i} = 0,
\end{equation}

где $L_{f_{x_i}} = \frac{\partial L}{\partial f_{x_i}}, \; i=\overline{1,n}$.

Вернёмся к исходной задаче. Подынтегральная функция
\begin{equation}
L = \left| \nabla f - \textbf{v} \right|^2 = \left( \frac{\partial f}{\partial x_1} - v_{x_1} \right)^2 + \left( \frac{\partial f}{\partial x_2} - v_{x_2} \right)^2 + \ldots + \left( \frac{\partial f}{\partial x_n} - v_{x_n} \right)^2 = 
\end{equation}
\begin{equation}
= \sum_{i=1}^n \left( \frac{\partial f}{\partial x_i} - v_{x_i} \right)^2
\end{equation}
 Вычислим соответствующие частные производные:
\begin{equation}
L_f = \frac{\partial L}{\partial f} = 0
\end{equation}
\begin{equation}
L_{f_{x_i}} = \frac{\partial L}{\partial \left(\frac{\partial f}{\partial x_i} \right)} = 2 \left(\frac{\partial f}{\partial x_i} - v_{x_i} \right), \; i=\overline{1,n}
\end{equation}


Тогда уравнение Эйлера-Лагранжа (\ref{Euler}) примет вид:
\begin{equation}
-2 \frac{\partial}{\partial x_1} \left( \frac{\partial f}{\partial x_1} - v_{x_1} \right) -2 \frac{\partial}{\partial x_2} \left( \frac{\partial f}{\partial x_2} - v_{x_2} \right) - \ldots -2 \frac{\partial}{\partial x_n} \left( \frac{\partial f}{\partial x_n} - v_{x_n} \right) = 0
\end{equation}

\begin{equation}
\left( \frac{\partial^2}{\partial x_1^2} + \frac{\partial^2}{\partial x_2^2} + \ldots + \frac{\partial^2}{\partial x_n^2} \right) f - \left( \frac{\partial v_{x_1}}{\partial x_1} + \frac{\partial v_{x_2}}{\partial x_2} + \ldots + \frac{\partial v_{x_n}}{\partial x_n} \right)= 0
\end{equation}

Вводя оператор Лапласа $$\Delta = \sum_{i=1}^n \frac{\partial^2}{\partial x_i^2}$$ и оператор дивергенции $$div \textbf{v} = \sum_{i=1}^n \frac{\partial v_{x_i}}{\partial x_i}$$ и учитывая полученное соотношение, исходную задачу (\ref{min})-(\ref{dirichle}) можно привести к виду:
\begin{align}
\Delta f = div v \label{poisson}\\
f|_{\delta\Omega} = f^*_{\delta\Omega} \label{dirichle2}
\end{align}

Задача (\ref{poisson})-(\ref{dirichle2}) является уравнением Пуассона с граничными условиями Дирихле. %Далее данная задача дискретизируется и решается численными методами.

\section{Приближение методом конечных разностей}

Вариационная задача (\ref{min})-(\ref{dirichle}) и соответствующее ей уравнение Пуассона (\ref{poisson}) с граничными условиями Дирихле (\ref{dirichle2}) могут быть дискретезированы и решены методом конечных разностей.

Пусть $S,\Omega$ — множества, состоящие из конечного числа точек, определённые на бесконечной дискретной сетке. Отметим, что $S$ может включать все точки карты признака или только часть их них. Для каждого пикселя $p\in S$ определим $2n$ направлений минимизации. Обозначим через $N_p$ множество ближайших соседей по этим направлениям из множества $S$, число которых ограничено $2n$, а также через $\langle p,q \rangle$ обозначим пару точек, где $q \in N_p$. Граница $\Omega$ есть $\delta \Omega = \lbrace p \in S \backslash \Omega : N_p \cap \Omega \neq 0 \rbrace$. $f_p$ — значение $f$ в точке $p$. Тогда задача сводится к расчёту набора интенсивностей $f|_\Omega = \lbrace f_p, p \in \Omega \rbrace$.

Для дискретизации задачи (\ref{min})-(\ref{dirichle}) воспользуемся односторонней аппроксимацией градиента по каждому из направлений минимизации. После подстановки получаем:
\begin{equation}
\begin{cases}
	\min_{f|_\Omega} \sum_{\langle p,q \rangle \cap \Omega \neq 0} \left( f_p - f_q - v_{pq} \right)^2 \\
	f_p = f^*_q, \; \forall p \in \delta \Omega,
\end{cases}
\end{equation}
где $v_{pq}$ — проекция $\textbf{v} \left(\frac{p+q}{2} \right)$ на направление $(p,q)$.

Дискретизация задачи (\ref{poisson}) использует аппроксимацию оператора Лапласа по схеме "крест":

\begin{equation}
\Delta f = \frac{1}{h^2}\sum_{q \in N_p \cap \Omega} \left(f_p - f_q \right) = \frac{1}{h^2} \left( |N_p|f_p - \sum_{q \in N_p \cap \Omega} f_q \right),
\end{equation}
где шаг сетки $h=1$ по всем направлениям, т.к. он соответствует расстоянию между пикселями; $|N_p|$ — число ближайших соседей точки $p$ или, что то же самое, число элементов множества $N_p \cap \Omega$. Для двумерных изображений $|N_p| = 4$ во внутренних точках области $\Omega$ и $N_p<4$ — в точках на границе $S$.

После подстановки аппроксимации оператора Лапласа в задачу (\ref{poisson}) и учёта граничных условий (\ref{dirichle2}) получаем систему уравнений:
\begin{equation}
|N_p|f_p - \sum_{q \in N_p \cap \Omega}f_q = \sum_{q \in N_p \cap \delta \Omega} f^*_q + \sum_{q \in N_p} v_{pq}, \; \forall p \in \Omega
\end{equation}

В тех случаях, когда точка $p$ — внутренняя точка области $\Omega$, то есть $N_p$ целиком содержится в $\Omega$, в правой части уравнения не содержится граничных условий и оно превращается в:
\begin{equation}
|N_p|f_p - \sum_{q \in N_p \cap \Omega}f_q = \sum_{q \in N_p} v_{pq}
\end{equation}

Полученная система является классической системой линейных уравнений вида $Ax = b$. Матрица $A$ является сильно разряженной, симметричной, трёхдиагональной, положительно определённой.

\section{Выбор параметров задачи. Вектор направлений}
\label{sec:vec}

В задаче минимизации (\ref{min}) и уравнении Пуассона (\ref{poisson}) присутствует параметр $\textbf{v}$ — так называемый вектор направлений \cite{Perez}\cite{Matias}. Рассмотрим его более подробно.

Вспомним, что рассматриваемая задача минимизации выглядит следующим образом:
\begin{equation}\label{minv}
\min_f \int\int_\Omega |\nabla f-\textbf{v}|^2
\end{equation}
\begin{equation}\label{dirichlev}
f|_{\delta\Omega} = f^*_{\delta\Omega},
\end{equation}

Простейший выбор $\textbf{v}$ состоит в том, чтобы выбрать его равным градиенту одной из накладываемых карт:
\begin{equation}
\textbf{v} = \nabla g,
\end{equation}
где $g$ — одно из изображений.

Подставляя в (\ref{minv}), получаем:
\begin{equation}
\min_f \int\int_\Omega |\nabla f-\textbf{v}|^2 = \min_f \int\int_\Omega |\nabla f-\nabla g|^2
\end{equation}

{\LARGE\textit{подстановка v в уравнение Пуассона и разностную схему...}}

Данная задача представляет собой задачу о минимизации разности градиентов искомой карты признаков и одной из накладываемых. Таким образом, решение задачи будет иметь градиент, максимально схожий с градиентом одной из данных карт. Тем самым гарантируется гладкость получаемого изображение на стыке двух карт.

Однако такой подход учитывает градиент лишь одной из накладываемых карт. В случае, когда у одного из изображений на границе присутствуют некоторые характерные особенности, такой выбор $\textbf{v}$ может привести к нежелательным результатам.

Одним из подходов, учитывающих градиенты обеих карт признаков, является метод усреднённого градиента:
\begin{equation}
\textbf{v} = \frac{1}{2} \left( \nabla g_1 + \nabla g_2 \right),
\end{equation}
где $g_1$ и $g_2$ — первая и вторая карта признаков соответственно.

Взяв в качестве вектора направлений $\textbf{v}$ усреднённый градиент обоих изображений, мы сможем получить гладкий переход из одного изображения в другое.

Другим подходом, учитывающим особенности обеих карт признаков, является метод сохранения максимального значения градиента, где вектор направлений вводится следующим образом:
\begin{equation}
\textbf{v}(x) = \left\lbrace \begin{array}{lr}
\nabla g_1(x), & |g_1(x)| \geq |g_2(x)| \\
\nabla g_2(x), & |g_1(x)| < |g_2(x)|
\end{array} \right.
\end{equation}

\section{Склейка двумерных карт признаков}

Результат работы алгоритма для двумерных карт признаков представлен на Рис. \ref{target2d}-\ref{res2d}.

На Рис.\ref{target2d} изображены две карты признаков, расположенные рядом друг с другом с небольшим наложением одной поверх второй. Область пересечения обозначена красным цветом. Границу раздела двух изображений можно видеть невооружённым глазом.

\begin{figure}[htbp]
\begin{subfigure}{1.0\textwidth}
  \centering  
  \includegraphics[width=0.8\textwidth]{images/target.jpg}
  \caption{}\label{target2d}
\end{subfigure}

\begin{subfigure}{1.0\textwidth}
  \centering  
  \includegraphics[width=0.8\textwidth]{images/res.jpg}
  \caption{}\label{res2d}
\end{subfigure}
  \caption{Изображения, полученные путём а) — наложения без сглаживания; b) — наложения со сглаживанием.}
\end{figure}


На Рис.\ref{res2d} изображены те же самые карты признаков, но в области их пересечения был применён алгоритм сглаживания. Как можно видеть, граница раздела двух изображений была полностью стёрта. В результате мы имеем одну карту признака области большего размера, полученную из двух карт областей меньшего размера.

\section{Склейка трёхмерных карт признаков}

Предложенный алгоритм также может быть обобщён на трёхмерные карты признаков. Формулы (\ref{min})-(\ref{dirichle2}) справедливы и в том случае, когда области $S$ и $\Omega$ являются подмножествами $R^3$, а функции $f$ и $f^*$ определены на соответствующих трёхмерных областях. Отличия от соответствующей двумерной задачи возникают лишь на этапе её дискретизации и решении численными методами, учёт которых не представляет из себя существенной трудности.

На Рис.\ref{target3d}-\ref{res3d}
представлен результат работы алгоритма для трёхмерных карт признаков.

На рисунках изображены три различных среза карт признаков, так что видна внутренняя часть областей. Как мы видим на Рис.\ref{target3d}, в области пересечения можно наблюдать чётко выделенную границу раздела двух карт признаков. Однако в результате сглаживания (Рис.\ref{res3d}) граница раздела исчезает и становится незаметной, причём как на поверхности, так и внутри карты. Таким образом, в результате мы имеем одну карту признака области большего размера, полученную из двух карт трёхмерных областей меньшего размера.
\begin{figure}[htbp]
\begin{subfigure}{1.0\textwidth}
  \centering  
  \includegraphics[width=0.8\textwidth]{images/slicesnoblending.png}
  \caption{}\label{target3d}
\end{subfigure}

\begin{subfigure}{1.0\textwidth}
  \centering  
  \includegraphics[width=0.8\textwidth]{images/slices.png}
  \caption{}\label{res3d}
\end{subfigure}
  \caption{Срезы трёхмерных карт признаков, полученные путём а) — наложения без сглаживания; b) — наложения со сглаживанием.}
\end{figure}

\section{Заключение}
В результате данной работы был разработан инструмент, позволяющий состыковать несколько карт признаков (дву- или трёхмерных), чтобы получить одну карту объекта большего размера.

В дальнейшем, имея данный инструмент, можно будет приступить непосредственно к решению проблемы ремасштабирования карт признаков.

\begin{thebibliography}{9}
\bibitem{Perez}P. Perez, M. Gangnet, and A. Blake, Poisson image editing, ACM Transactions on Graphics, 22 (2003), p. 313.
\bibitem{Matias}J. Matías Di Martino, Gabriele Facciolo, and Enric Meinhardt-Llopis, Poisson Image Editing, Image Processing On Line, 6 (2016), pp. 300–325.
\bibitem{Springer}  E. Z. Nonlinear functional analysis and its applications: Variational methods and optimization.
— Springer, 1985. — P. 662.
\end{thebibliography}

\end{document}
