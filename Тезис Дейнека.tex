\documentclass[12pt,a4paper]{scrartcl}
\usepackage[utf8]{inputenc}
\usepackage{fullpage}
\usepackage[english,russian]{babel}
\usepackage{indentfirst}
\usepackage{misccorr}
\usepackage{graphicx}
\usepackage{amsmath}
\usepackage{wrapfig}
\usepackage{float}


\begin{document}

На сегодняшний день в некоторых прикладных областях, таких как геологическая разведка и нефтедобыча, существует проблема предсказания некоторого свойства объекта в пространстве по известной точной информации о его локальной структуре — карте признака. Примером таких карт может послужить карта радиоактивных элементов в горной породе. Одной из подзадач данной проблемы является ремасштабирование карт признаков.
При переходе на большие масштабы часто возникает ситуация, когда необходимо состыковать несколько карт признаков, чтобы получить одну карту объекта большего размера. Одним из подходов для получения карт является их генерация с помощью нейросетевых методов — в этом случае различные карты признаков получаются независимо друг от друга, из-за чего при их стыковке можно наблюдать резкую границу между ними. Для решения данной проблемы ставится задача об интерполяции двух функций в области их перекрытия или, иначе говоря, "сглаживание" двух изображений в месте их стыковки.  
В данной работе эта задача формулируется как задача о минимизации некоторого функционала, которая в последствии сводится к уравнению Пуассона с граничными условиями Дирихле. Затем аналогичная задача ставится уже в случае трёхмерных карт признаков и анализируются различия в постановке граничных условий в двух- и трёхмерном случаях.


\end{document}